\documentclass[conference]{IEEEtran}
\IEEEoverridecommandlockouts
% The preceding line is only needed to identify funding in the first footnote. If that is unneeded, please comment it out.
\usepackage{cite}
\usepackage{amsmath,amssymb,amsfonts}
\usepackage{algorithmic}
\usepackage{graphicx}
\usepackage{textcomp}
\usepackage{xcolor}
\usepackage{soul}
\def\BibTeX{{\rm B\kern-.05em{\sc i\kern-.025em b}\kern-.08em
    T\kern-.1667em\lower.7ex\hbox{E}\kern-.125emX}}
\usepackage{hyperref}
\hypersetup{
	citecolor = blue,
	linkcolor = blue,
	urlcolor = blue,
	colorlinks = true,
	linkbordercolor = {white}
}


\begin{document}

\title{\textit{CoronaZ}: another distributed systems project
\\{\Large Simulating a contact tracing application in a scalable environment}
%\thanks{Identify applicable funding agency here. If none, delete this.}
}

\author{
	
	\IEEEauthorblockN{Stefan Ciprian Voinea}
	\IEEEauthorblockA{Università degli Studi di Padova}
	\textit{stefanciprian.voinea@studenti.unipd.it}
	\and
	\IEEEauthorblockN{Stefan Vladov}
	\IEEEauthorblockA{University}
	\textit{mail}
	\and
	\IEEEauthorblockN{Fabian Rensing}
	\IEEEauthorblockA{University}
	\textit{mail}
}

\maketitle
\thispagestyle{plain}
\pagestyle{plain}

%\begin{abstract}
%This document is a model and instructions for \LaTeX.
%This and the IEEEtran.cls file define the components of your paper [title, text, heads, etc.]. *CRITICAL: Do Not Use Symbols, Special Characters, Footnotes, 
%or Math in Paper Title or Abstract.
%\end{abstract}

%\begin{IEEEkeywords}
%component, formatting, style, styling, insert
%\end{IEEEkeywords}

\section{Introduction}\label{sec:introduction}

	This brief paper describes \textit{CoronaZ}, a project for the Distributed Systems course at the \textit{University of Helsinki}.
	
	All the code of the project is publicly available on GitHub repository\cite{coronaz_repo}.
	
	This project simulates a contact tracing application where each node represents a person (or a unique device attached to someone) that send signals to each other when in range and communicate the data collected to a server using the \textit{publish-subscribe} pattern.
	The server, called \textit{broker}, can then be polled by a node called \textit{consumer} that will send the data to a database.
	A front-end application then requests this data and displays the movement and the latest updates via the browser.
	
	The idea came from simulating this kind of movements with Arduino boards capable of communicating between them using the \textit{nrf24l01} and to the broker with \textit{esp8266}.
	Unfortunately this was not possible given the relatively strict amount of time that each of the students involved could dedicate to the project and the waiting time to get the necessary hardware.

\section{Technological choices}\label{sec:technological_choices}

	In this section we describe and explain why we have decided to use certain technologies rather than others.
	
	\begin{itemize}
		\item \textit{Programming language}: we chose Python because it is a fast programming language for creating prototypes. Also all group components are fluent with Python;
		\item \textit{MQTT Broker}: we chose \textit{Apache Kafka} as broker for our project since it is one of the most used brokers in the market and it is has a large community that supports the project. Apache Kafka also handles well scaling and integration with other systems;
		\item \textit{Containers}: we chose to use \textit{Docker} and \textit{docker-compose} given the easiness of constructing and spawning nodes. As explained in \ref{sec:architecture}, we have a \texttt{docker-compose.yml} that handles that base components such as the broker and the database;
		\item \textit{Database}: we chose \textit{MongoDB} for the scalability and the ability to adapt to new data formats. Also because some group components had previous experience with it;
		\item \textit{Back-end}: we chose \textit{NodeJS} since, like for the database, some group components had previous experience with it.
		\item \textit{Front-end}: 
	\end{itemize}

	\subsection{System requirements}
	
		The system requirements for this project are simple:
		\begin{itemize}
			\item \textit{Docker} and \textit{docker-compose};
			\item Ubuntu or another Linux system with \texttt{jq} installed for starting the project using the \textit{init-project.sh} bash script;
		\end{itemize}

\section{Architecture}\label{sec:architecture}
	
	We can divide our project in four major areas: \textit{nodes}, \textit{broker}, \textit{DB consumer}, \textit{database}, \textit{front-end}.

	\begin{figure}[htbp]
		\centerline{\includegraphics[width=\linewidth]{img/coronaz.png}}
		\caption{Architecture of the \textit{CoronaZ} project.}
		\label{fig:architecture}
	\end{figure}
	
	\subsection{Nodes}
	
		This can be considered a single component since each node can be spawned separately from one another.
		
		\hl{When each node spawns in the map it is placed on a random position.
		Every second the node will ``move''.
		When the node moves}
		
		\hl{Nodes introduced into the simulation can be \textit{infected} or \textit{non-infected} (or \textit{safe}).
		When a node is infected it will keep steady and will not move for a certain amount of time declared in the parameter \texttt{infection\_cooldown}, which defines the number of seconds that the node will stay in place.}
		
		In our simulation the nodes can all connect to each other since they are all in the same network and each of them can hear the data sent in broadcast by the others.
		In a more realistic situation nodes would be only capable of hearing the signals of nodes nearby them, like in Fig.\ref{fig:architecture}~where node \textit{A} can communicate only with node \textit{B} and nodes \textit{B}, \textit{C} and \textit{D} are nearby each other enough for them to hear their signal.

		\hl{Here is an example of a message that is sent in broadcast from a node:}
		\begin{verbatim}
{  
   "uuid":"ff0a1bda-34b9-11eb-b339 ... ",
   "position":[1, 5],
   "infected":false,
   "timestamp":"2020-12-02 16:19 ... ",
   "alive":true
}  
		\end{verbatim}
		
		Each node has a unique \textit{UUID} that is generated in Python, using the \texttt{uuid}\cite{uuid}.
		This is created via a combination of \texttt{MAC} address and the \texttt{IP} address of the machine the script runs on and the timestamp on when the process starts.

		\hl{threads}

	\subsection{MQTT broker}	
	
		Apache Kafka is composed by the Kafka container and the Zookeeper container.
		Kafka is an important part in the project since it is the \textit{broker} in the \textit{publish-subscribe} pattern.
		
		Each node, after it has collected the IDs of the other nearby it, will send the list ot these IDs, with other information, to the broker.
		The \textit{topic}, in our code, used by the nodes is ``\textit{coronaz}''.
		This will contain all the information send by the nodes to the broker, which will later forward them to the consumer when it asks for them.
		
		\hl{Here is an example of a message that is sent to the broker:}
		\begin{verbatim}
tmp
tmp
tmp
		\end{verbatim}
		
	
	\subsection{DB consumer}
	
		The DB consumer can be considered as a single entity since it is independent both from Kafka and from Mongo.
		The consumer is subscribed to the Kafka topic that contains the new messages from the nodes, in our case ``\textit{coronaz}''.
		When a new message arrives, the consumer gets it and, every ten messages, it aggregates them in a \texttt{json} that will be sent to the MongoDB database.
	
	\subsection{Database}
	
		As the other components, the database runs in a Docker container.
		
		It accepts and stores incoming data from the Consumer.
		The database is accessible from the front-end, which will request data every second in order to always display fresh information.
	
		\begin{figure}[htbp]
			\centerline{\includegraphics[width=\linewidth]{img/database.png}}
			\caption{CoronaZ database structure.}
			\label{fig:database}
		\end{figure}
	
	\subsection{Back-end}
	
		For the back-end of the project, \textit{NodeJS} does \hl{...}
	
	\subsection{Front-end}

		The front-end of the project, programmed using \textit{React}, shows the evolution of the system and the simulations.
		It shows the movement of the nodes in the map, along with which one has been infected or not.
		
		\begin{figure}[htbp]
			\centerline{\includegraphics[width=\linewidth]{img/frontend.png}}
			\caption{CoronaZ front-end.}
			\label{fig:front-end}
		\end{figure}
	
\section{Docker networking}

	Since this project massively uses Docker containers, there is a docker network underneath it that connects all the components.
	
\section{Scalability and fault tolerance}

	We have tested the scalability and the fault tolerance of the project in the following ways:
	
	\begin{itemize}
		
		\item \hl{\textit{unexpectedly shutting down the MongoDB database}:}
		
		\item \hl{\textit{unexpectedly shutting down the consumer}:}
		
		\item \hl{\textit{unexpectedly shutting down the front end}:}
		
		\item \hl{\textit{unexpectedly shutting down the broker}:}
		
		\item \textit{adding more nodes}: when a node is added it will start communicating with the other nodes already in the system and it will start sending messages to the broker.
		
	\end{itemize}

	In general for the errors that can be felt in the front-end, such as the database not responding or the back-end server not responding, the user does not really need to know what exact component went down.
	This is also for security reasons in case a \textit{malicious} user wants to find the vulnerabilities; without the specific logs it is harder for him to find the exact breaking point of the project.
	
\section{Simulation}\label{sec:simulation}

	To start the project we have made a \textit{init-project.sh} script that asks the parameters with which the simulation will take place, that creates the docker network, executes the \texttt{docker-compose} (\texttt{up} and \texttt{down}) commands and manages (and spawns) the number of nodes.
	
	\begin{figure}[htbp]
		\centerline{\includegraphics[width=\linewidth]{img/script.png}}
		\caption{CoronaZ front-end.}
		\label{fig:frontend}
	\end{figure}

	The script will ask for the run parameters and will set them on the file, otherwise will run with a set of defaults.
	The default parameters for the run are (in \textit{json} format):
	\begin{verbatim}
    {
        "field_width" : 100,
        "field_height" : 100,
        "scale_factor" : 5,
        "zombie_lifetime" : 120,
        "infection_radius" : 2,
        "infection_cooldown" : 15
    }
	\end{verbatim}
	
	These parameters stand for:
	\begin{itemize}
		\item ``\textit{field\_width}'': width of the map;
		\item ``\textit{field\_height}'': height of the map;
		\item ``\textit{scale\_factor}'':
		\item ``\textit{zombie\_lifetime}'': lifetime in seconds of the zombies;
		\item ``\textit{infection\_radius}'':
		\item ``\textit{infection\_cooldown}'': ``\textit{cooldown}'' period in seconds in which the nodes stand in order
	\end{itemize}
	
	\begin{figure}[htbp]
		\centerline{\includegraphics[width=\linewidth]{img/sim_1.png}}
		\caption{\hl{CoronaZ simulation at time X.}}
		\label{fig:sim_1}
	\end{figure}
	\begin{figure}[htbp]
		\centerline{\includegraphics[width=\linewidth]{img/sim_2.png}}
		\caption{\hl{CoronaZ simulation at time X+Y.}}
		\label{fig:sim_2}
	\end{figure}
	\begin{figure}[htbp]
		\centerline{\includegraphics[width=\linewidth]{img/sim_3.png}}
		\caption{\hl{CoronaZ simulation at time X+Y+Z.}}
		\label{fig:sim_3}
	\end{figure}

\section{Future work}\label{sec:future_work}

	Given the nature of the project there won't be future releases but in this section we want to talk about what can be done to improve the project.
	
	There are various improvements that can be done to make CoronaZ a more interesting and stable simulator, we will tackle them based on the areas defined in \ref{sec:architecture}.
	
	\hl{IMPROVEMENTS}
	
\section{Conclusions}\label{sec:conclusions}

	This project represents a simulation of a contact tracing application where each node that spawns in out network communicates its position to the others in range and will send the data collected to a central server with the \textit{publish-subscribe} pattern.

\begin{thebibliography}{00}
	
	\bibitem{coronaz_repo}
		\textit{CoronaZ repository:}\\
		\url{https://github.com/cipz/CoronaZ/}
	
	\bibitem{uuid}
		\textit{\texttt{uuid} — UUID objects according to RFC 4122:}\\
		\url{https://docs.python.org/3.8/library/uuid.html}
		
\end{thebibliography}

\end{document}
